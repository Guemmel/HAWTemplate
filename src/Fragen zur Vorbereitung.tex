\section{Fragen zur Vorbereitung}
\subsection{Wie lauten die Kirchhoffschen Gesetze?}
Es gibt zwei Kirchhoffsche Gesetze. Das erste, die Knotenpunktsatz, besagt, dass in jedem Knoten eines Stromkreises, die zufließenden Ströme gleich den abfließenden Strömen ($I_zu = I_ab$) sind. Ein Knoten meint dabei einen Verzweigungspunkt im Stromkreis.

Das zweite Kichhoffsche Gesetz beschreibt das Verhalten von Spannungen innerhalb einer Masche zueinander. Es wird daher auch Maschensatz genannt. Nach dem Maschensatz ist die vorzeichenbehaftete Summe aller Spannungen innerhalb eine Masche gleich 0 ($U_q - \sum_{0}^{n} U_n = 0$)