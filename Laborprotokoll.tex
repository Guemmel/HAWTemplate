% !TeX encoding = uft8

\documentclass[11pt, 
                toc=listof, 
                toc=bibliography, 
                footnotes=multiple, 
                parskip=half, 
                numbers=noendperiod,
                ngerman
            ]{scrartcl}
\usepackage{pdfpages}
\usepackage{pdflscape}
\usepackage[
    bookmarks,
    bookmarksnumbered,
    bookmarksopen=true,
    bookmarksopenlevel=1,
    colorlinks=true,
% diese Farbdefinitionen zeichnen Links im PDF farblich aus
    linkcolor=HAW,
    anchorcolor=HAW,% Ankertext
    citecolor=HAW, % Verweise auf Literaturverzeichniseinträge im Text
    filecolor=HAW, % Verknüpfungen, die lokale Dateien öffnen
    menucolor=HAW, % Acrobat-Menüpunkte
    urlcolor=HAW,
% diese Farbdefinitionen sollten für den Druck verwendet werden (alles schwarz)
    %linkcolor=black, % einfache interne Verknüpfungen
    %anchorcolor=black, % Ankertext
    %citecolor=black, % Verweise auf Literaturverzeichniseinträge im Text
    %filecolor=black, % Verknüpfungen, die lokale Dateien öffnen
    %menucolor=black, % Acrobat-Menüpunkte
    %urlcolor=black,
%
    backref, % Quellen werden zurück auf ihre Zitate verlinkt
    % pdftex,
    % plainpages=false, % zur korrekten Erstellung der Bookmarks
    % pdfpagelabels=true, % zur korrekten Erstellung der Bookmarks
    % hypertexnames=false, % zur korrekten Erstellung der Bookmarks
    % linkcolor=black,
    linktoc=all,
    pdfpagelabels=true
]{hyperref}
\usepackage{caption}
% Befehle, die Umlaute ausgeben, führen zu Fehlern, wenn sie hyperref als Optionen übergeben werden
% \hypersetup{
%     % pdftitle={\titel -- \untertitel},
%     % pdfauthor={\autorName},
%     % pdfcreator={\autorName},
%     % pdfsubject={\titel -- \untertitel},
%     % pdfkeywords={\titel -- \untertitel},
% }



\usepackage[utf8]{inputenc}
\usepackage[T1]{fontenc}
\usepackage[ngerman]{babel}
\usepackage{xcolor}
\usepackage{colortbl}

\usepackage{ulem}

\usepackage{relsize} % Releative Schrifgröße
\usepackage{lipsum}

\usepackage[titles]{tocloft} % Inhaltsverzeichnis

% Deckblatt
\usepackage{mwe}
\usepackage[absolute]{textpos}

\usepackage{setspace}
\usepackage{geometry}

\usepackage{graphicx}
\usepackage{graphics}
\usepackage{adjustbox}

\usepackage{float}
\restylefloat{table}

\usepackage{todonotes}


% Symbolverzeichnis
\usepackage{acronym}
\usepackage[intoc]{nomencl}
\let\abbrev\nomenclature
\renewcommand{\nomname}{Abkürzungsverzeichnis}
\setlength{\nomlabelwidth}{.25\hsize}
\renewcommand{\nomlabel}[1]{#1 \dotfill}
\setlength{\nomitemsep}{-\parsep}

\usepackage{varioref} % Elegantere Verweise. „auf der nächsten Seite“
\usepackage{url} % URL verlinken, lange URLs umbrechen etc.

\usepackage{chngcntr} % fortlaufendes Durchnummerieren der Fußnoten
% \usepackage[perpage]{footmisc} % Alternative: Nummerierung der Fußnoten auf jeder Seite neu

\usepackage{ifthen} % bei der Definition eigener Befehle benötigt
\usepackage[square]{natbib} % wichtig für korrekte Zitierweise

\usepackage{circuitikz}
\usepackage{pgfplots}
\pgfplotsset{width=10cm,compat=1.9}
\usepackage{csvsimple}
\usepackage{siunitx}
\usepackage{gensymb}
\usepackage{accents}

\sisetup{locale=DE,per-mode=symbol,range-phrase=--,range-units=single,product-units=single}
\title{Titel}
\subtitle{Untertitel}
\author{Author1 (), \\ Author2}
\date{\today{}}
\newcommand{\abgabetermin}{07.06.2024}
\newcommand{\labdate}{24.05.2024}
%%%%%%%%%%%%%%%%%%%%%%%%%%%%%%%%%%%%%%%%%%%%%%%%%%%%%%%%%%%%%%%%%%%%%%%
% SEITENLAYOUT
%%%%%%%%%%%%%%%%%%%%%%%%%%%%%%%%%%%%%%%%%%%%%%%%%%%%%%%%%%%%%%%%%%%%%%%

\setlength{\topskip}{\ht\strutbox} % behebt Warnung von geometry
\geometry{a4paper,left=25mm,right=25mm,top=25mm,bottom=35mm}

%%%%%%%%%%%%%%%%%%%%%%%%%%%%%%%%%%%%%%%%%%%%%%%%%%%%%%%%%%%%%%%%%%%%%%%
% DRUCKMODUS
%%%%%%%%%%%%%%%%%%%%%%%%%%%%%%%%%%%%%%%%%%%%%%%%%%%%%%%%%%%%%%%%%%%%%%%

% Schalter für Druckmodus: \printmodetrue für S/W-Druck, \printmodefalse für farbige Links
\newif\ifprintmode
\printmodefalse % Standardmäßig farbige Links (auf 'true' setzen für S/W-Druck)

%%%%%%%%%%%%%%%%%%%%%%%%%%%%%%%%%%%%%%%%%%%%%%%%%%%%%%%%%%%%%%%%%%%%%%%
% SCHRIFTARTEN UND TEXTSTILE
%%%%%%%%%%%%%%%%%%%%%%%%%%%%%%%%%%%%%%%%%%%%%%%%%%%%%%%%%%%%%%%%%%%%%%%

\usepackage{eurosym}
\usepackage{textcomp}
\usepackage{lmodern}
\usepackage{relsize}

% Absatz- und Zeilenumbrüche verbessern
\clubpenalty = 10000
\widowpenalty = 10000
\displaywidowpenalty = 10000
\onehalfspacing

%%%%%%%%%%%%%%%%%%%%%%%%%%%%%%%%%%%%%%%%%%%%%%%%%%%%%%%%%%%%%%%%%%%%%%%
% KOPF- UND FUSSZEILENKONFIGURATION
%%%%%%%%%%%%%%%%%%%%%%%%%%%%%%%%%%%%%%%%%%%%%%%%%%%%%%%%%%%%%%%%%%%%%%%

\usepackage[
	automark, % Kapitelangaben in Kopfzeile automatisch erstellen
	headsepline, % Trennlinie unter Kopfzeile
	ilines % Trennlinie linksbündig ausrichten
]{scrlayer-scrpage}

\pagestyle{scrheadings}
\makeatletter
\chead{%
	\large{\textsc{\@title}}\\ 
	\small{\rightmark}}
\makeatother
\ohead{}
\setlength{\headheight}{15mm} % Höhe der Kopfzeile

% Fußzeile: links, mitte, rechts
\ifoot{Gomoll, Hinterthan}
\cfoot{\today}
\ofoot{Seite \pagemark}

%%%%%%%%%%%%%%%%%%%%%%%%%%%%%%%%%%%%%%%%%%%%%%%%%%%%%%%%%%%%%%%%%%%%%%%
% ÜBERSCHRIFTENFORMATIERUNG
%%%%%%%%%%%%%%%%%%%%%%%%%%%%%%%%%%%%%%%%%%%%%%%%%%%%%%%%%%%%%%%%%%%%%%%

\newcommand{\headingSpace}{1.5cm}
\renewcommand*{\othersectionlevelsformat}[3]{
  \makebox[\headingSpace][l]{#3\autodot}
}

% Für die Einrückung wird das Paket tocloft benötigt
\cftsetindents{section}{0.0cm}{\headingSpace}
\cftsetindents{subsection}{0.0cm}{\headingSpace}
\cftsetindents{subsubsection}{0.0cm}{\headingSpace}
\cftsetindents{figure}{0.0cm}{\headingSpace}
\cftsetindents{table}{0.0cm}{\headingSpace}

%%%%%%%%%%%%%%%%%%%%%%%%%%%%%%%%%%%%%%%%%%%%%%%%%%%%%%%%%%%%%%%%%%%%%%%
% FARBEN
%%%%%%%%%%%%%%%%%%%%%%%%%%%%%%%%%%%%%%%%%%%%%%%%%%%%%%%%%%%%%%%%%%%%%%%

% Farbdefinitionen
\definecolor{HAW}{HTML}{163b9a}
\definecolor{odd}{HTML}{b3b3b3}
\definecolor{heading}{rgb}{0.64,0.78,0.86}

% Link-Farben je nach Druckmodus
\ifprintmode
    % Schwarze Links für Druck
    \colorlet{LinkColor}{black}
\else
    % Farbige Links für Bildschirmanzeige
    \colorlet{LinkColor}{HAW}
\fi

%%%%%%%%%%%%%%%%%%%%%%%%%%%%%%%%%%%%%%%%%%%%%%%%%%%%%%%%%%%%%%%%%%%%%%%
% NUMMERIERUNG UND ZÄHLER
%%%%%%%%%%%%%%%%%%%%%%%%%%%%%%%%%%%%%%%%%%%%%%%%%%%%%%%%%%%%%%%%%%%%%%%

\counterwithout{footnote}{section} % Fußnoten fortlaufend durchnummerieren
\setcounter{tocdepth}{3} % im Inhaltsverzeichnis werden die Kapitel bis zum Level der subsubsection übernommen
\setcounter{secnumdepth}{3} % Kapitel bis zum Level der subsubsection werden nummeriert

%%%%%%%%%%%%%%%%%%%%%%%%%%%%%%%%%%%%%%%%%%%%%%%%%%%%%%%%%%%%%%%%%%%%%%%
% AUFZÄHLUNGSFORMATIERUNG
%%%%%%%%%%%%%%%%%%%%%%%%%%%%%%%%%%%%%%%%%%%%%%%%%%%%%%%%%%%%%%%%%%%%%%%

\renewcommand{\labelenumi}{\arabic{enumi}.}
\renewcommand{\labelenumii}{\arabic{enumi}.\arabic{enumii}.}
\renewcommand{\labelenumiii}{\arabic{enumi}.\arabic{enumii}.\arabic{enumiii}}
\input{common/Befehle.tex}
\begin{document}

% Deckblatt

\include{src/Deckblatt}


% Inhalsverzeichnisse

\pagenumbering{roman}


\begingroup
\hypersetup{linkcolor=black}
\tableofcontents
\clearpage

\listoffigures


\listoftables
\clearpage

% \newcommand{\abkvz}{Abkürzungsverzeichnis}
% \renewcommand{\nomname}{\abkvz}
% \section*{\abkvz}
% \markboth{\abkvz}{\abkvz}
% \addcontentsline{toc}{section}{\abkvz}
% \input{common/Abkuerzungen.tex}
% \clearpage


\endgroup
% Inhalt

\pagenumbering{arabic}
\setcounter{page}{1}

\section{Materialliste}
In der folgenden \reftable{MateriallisteGesamt} sind für alle Laborversuche notwendigen Geräte hinterlegt. 
Die für die jeweiligen Aufgaben notwendigen Geräte sind zu Anfang einer jeden Aufgabe erwähnt.
\tabelle{Materialliste Gesamt}{MateriallisteGesamt}{MateriallisteGesamt.tex}

\section{Fragen zur Vorbereitung}
\subsection{Wie lauten die Kirchhoffschen Gesetze?}
Es gibt zwei Kirchhoffsche Gesetze. Das erste, die Knotenpunktsatz, besagt, dass in jedem Knoten eines Stromkreises, die zufließenden Ströme gleich den abfließenden Strömen ($I_zu = I_ab$) sind. Ein Knoten meint dabei einen Verzweigungspunkt im Stromkreis.

Das zweite Kichhoffsche Gesetz beschreibt das Verhalten von Spannungen innerhalb einer Masche zueinander. Es wird daher auch Maschensatz genannt. Nach dem Maschensatz ist die vorzeichenbehaftete Summe aller Spannungen innerhalb eine Masche gleich 0 ($U_q - \sum_{0}^{n} U_n = 0$)
\section{Laborversuche}


\subsection{Aufgabe 1}
\label{sctn:Aufgabe1}


Dafür werden die in \reftable{A1_Materialliste} aufgeführten Gegenstände benötigt.

\tabelle{In Aufgabe 1 verwendete Materialien}{A1_Materialliste}{Aufgabe1/Tables/Materialliste.tex}
% Hinleitung zur Aufgabe und zur Vorbereitung


% Vorberechnung
\tabelle{---}{A1_Berechnung}{Aufgabe1/Tables/Vorberechnung.tex}


% Messschaltung

\circuit{Messschaltung Kondensatoren in Parallelschaltung}{A1_Messschaltung}{Aufgabe1/Circuits/Circuit.tex}

% Messwerten

\tabelle{---}{A1_Messung}{Aufgabe1/Tables/Messung.tex}


% Auswertung mit Beantwortung der Fragen



% Einfügen einer Frage mit \Frage{}
\subsection{Aufgabe 2}
\label{sctn:Aufgabe2}


Dafür werden die in \reftable{A2_Materialliste} aufgeführten Gegenstände benötigt.

\tabelle{In Aufgabe 2 verwendete Materialien}{A2_Materialliste}{Aufgabe2/Tables/Materialliste.tex}
% Hinleitung zur Aufgabe und zur Vorbereitung


% Vorberechnung
\tabelle{---}{A2_Berechnung}{Aufgabe2/Tables/Vorberechnung.tex}


% Messschaltung

\circuit{Messschaltung Kondensatoren in Parallelschaltung}{A2_Messschaltung}{Aufgabe2/Circuits/Circuit.tex}

% Messwerten

\tabelle{---}{A2_Messung}{Aufgabe2/Tables/Messung.tex}


% Auswertung mit Beantwortung der Fragen



% Einfügen einer Frage mit \Frage{}
\subsection{Aufgabe 3}
\label{sctn:Aufgabe3}


Dafür werden die in \reftable{A3_Materialliste} aufgeführten Gegenstände benötigt.

\tabelle{In Aufgabe 3 verwendete Materialien}{A3_Materialliste}{Aufgabe3/Tables/Materialliste.tex}
% Hinleitung zur Aufgabe und zur Vorbereitung


% Vorberechnung
\tabelle{---}{A3_Berechnung}{Aufgabe3/Tables/Vorberechnung.tex}


% Messschaltung

\circuit{Messschaltung Kondensatoren in Parallelschaltung}{A3_Messschaltung}{Aufgabe3/Circuits/Circuit.tex}

% Messwerten

\tabelle{---}{A3_Messung}{Aufgabe3/Tables/Messung.tex}


% Auswertung mit Beantwortung der Fragen



% Einfügen einer Frage mit \Frage{}
\subsection{Aufgabe 4}
\label{sctn:Aufgabe4}


Dafür werden die in \reftable{A4_Materialliste} aufgeführten Gegenstände benötigt.

\tabelle{In Aufgabe 4 verwendete Materialien}{A4_Materialliste}{Aufgabe4/Tables/Materialliste.tex}
% Hinleitung zur Aufgabe und zur Vorbereitung


% Vorberechnung
\tabelle{---}{A4_Berechnung}{Aufgabe4/Tables/Vorberechnung.tex}


% Messschaltung

\circuit{Messschaltung Kondensatoren in Parallelschaltung}{A4_Messschaltung}{Aufgabe4/Circuits/Circuit.tex}

% Messwerten

\tabelle{---}{A4_Messung}{Aufgabe4/Tables/Messung.tex}


% Auswertung mit Beantwortung der Fragen



% Einfügen einer Frage mit \Frage{}


\clearpage
\nocite{*}


% % !TeX root = ..\..\Projektdokumentation.tex

\appendix
\pagenumbering{roman}


\section{Anhang}

\subsection{Anhang 1}
\label{A:Anhang1}
\lipsum[1]
\clearpage

\end{document}