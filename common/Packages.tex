\usepackage{pdfpages}
\usepackage{pdflscape}
\usepackage[
    bookmarks,
    bookmarksnumbered,
    bookmarksopen=true,
    bookmarksopenlevel=1,
    colorlinks=true,
% diese Farbdefinitionen zeichnen Links im PDF farblich aus
    linkcolor=SKM,
    anchorcolor=SKM,% Ankertext
    citecolor=SKM, % Verweise auf Literaturverzeichniseinträge im Text
    filecolor=SKM, % Verknüpfungen, die lokale Dateien öffnen
    menucolor=SKM, % Acrobat-Menüpunkte
    urlcolor=SKM,
% diese Farbdefinitionen sollten für den Druck verwendet werden (alles schwarz)
    %linkcolor=black, % einfache interne Verknüpfungen
    %anchorcolor=black, % Ankertext
    %citecolor=black, % Verweise auf Literaturverzeichniseinträge im Text
    %filecolor=black, % Verknüpfungen, die lokale Dateien öffnen
    %menucolor=black, % Acrobat-Menüpunkte
    %urlcolor=black,
%
    backref, % Quellen werden zurück auf ihre Zitate verlinkt
    % pdftex,
    % plainpages=false, % zur korrekten Erstellung der Bookmarks
    % pdfpagelabels=true, % zur korrekten Erstellung der Bookmarks
    % hypertexnames=false, % zur korrekten Erstellung der Bookmarks
    % linkcolor=black,
    linktoc=all,
    pdfpagelabels=true
]{hyperref}
\usepackage{caption}
% Befehle, die Umlaute ausgeben, führen zu Fehlern, wenn sie hyperref als Optionen übergeben werden
% \hypersetup{
%     % pdftitle={\titel -- \untertitel},
%     % pdfauthor={\autorName},
%     % pdfcreator={\autorName},
%     % pdfsubject={\titel -- \untertitel},
%     % pdfkeywords={\titel -- \untertitel},
% }



\usepackage[utf8]{inputenc}
\usepackage[T1]{fontenc}
\usepackage[ngerman]{babel}
\usepackage{xcolor}
\usepackage{colortbl}

\usepackage{ulem}

\usepackage{relsize} % Releative Schrifgröße
\usepackage{lipsum}

\usepackage[titles]{tocloft} % Inhaltsverzeichnis

% Deckblatt
\usepackage{mwe}
\usepackage[absolute]{textpos}

\usepackage{setspace}
\usepackage{geometry}

\usepackage{graphicx}
\usepackage{graphics}
\usepackage{adjustbox}

\usepackage{float}
\restylefloat{table}

\usepackage{todonotes}


% Symbolverzeichnis
\usepackage{acronym}
\usepackage[intoc]{nomencl}
\let\abbrev\nomenclature
\renewcommand{\nomname}{Abkürzungsverzeichnis}
\setlength{\nomlabelwidth}{.25\hsize}
\renewcommand{\nomlabel}[1]{#1 \dotfill}
\setlength{\nomitemsep}{-\parsep}

\usepackage{varioref} % Elegantere Verweise. „auf der nächsten Seite“
\usepackage{url} % URL verlinken, lange URLs umbrechen etc.

\usepackage{chngcntr} % fortlaufendes Durchnummerieren der Fußnoten
% \usepackage[perpage]{footmisc} % Alternative: Nummerierung der Fußnoten auf jeder Seite neu

\usepackage{ifthen} % bei der Definition eigener Befehle benötigt
\usepackage[square]{natbib} % wichtig für korrekte Zitierweise

\usepackage{circuitikz}
\usepackage{pgfplots}
% \pgfplotsset{width=10cm,compat=1.9}
\usepackage{bodeplot} % Provides formulas for plotting filter graphs
\usepackage{csvsimple}
\usepackage{siunitx}
\usepackage{gensymb}
\usepackage{accents}
\usepackage{changelog}

\sisetup{locale=DE,per-mode=symbol,range-phrase=--,range-units=single,product-units=single}