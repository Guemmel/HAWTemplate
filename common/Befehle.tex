\newcommand{\Anhang}[1]{\appendixname{}~\ref{A:#1}: \nameref{A:#1} auf Seite \pageref{A:#1}}
\newcommand{\AnhangKurz}[1]{\appendixname{}~\ref{A:#1}: \nameref{A:#1}}
\newcommand{\mysctnref}[1]{Abschnitt \ref{sctn:#1} (\nameref{sctn:#1})}
\newcommand{\mypageref}[1]{\ref{sctn:#1}: \nameref{sctn:#1} auf Seite \pageref{sctn:#1}}
\newcommand{\reftable}[1]{Tabelle \ref{tab:#1}: \nameref{tab:#1}}
\newcommand{\reftablefull}[1]{Tabelle \ref{tab:#1} auf Seite \pageref{tab:#1}}
\newcommand{\mycircuitref}[1]{Abbildung \ref{circuit:#1}: \nameref{circuit:#1}}
\newcommand{\myfigureref}[1]{Abbildung \ref{#1}: \nameref{#1}}
\newcommand{\myfigurereffull}[1]{Abbildung \ref{#1} auf Seite \pageref{#1}}
\newcommand{\mygraphref}[1]{Abbildung \ref{graph:#1}: \nameref{graph:#1}}
\newcommand{\mygraphreffull}[1]{Abbildung \ref{graph:#1}: \nameref{graph:#1} auf Seite \pageref{graph:#1}}

% fügt Tabellen aus einer TEX-Datei ein
\newcommand{\tabelle}[3] % Parameter: caption, label, file
    {\begin{table}[H]
    \centering
    \singlespacing
    \input{src/#3}
    \caption{#1}
    \label{tab:#2}
    \end{table}}
    
\newcommand{\tabelleAnhang}[2] % Parameter: file
    {\begin{table}[H]
    \centering
    \singlespacing
    \input{src/Tabellen/#2}
    \caption{#1}
    \end{table}}


\newcommand{\graph}[3]{
        \begin{figure}[H]
            \centering
            \input{src/#3}
            \caption{#1}
            \label{graph:#2}
        \end{figure}
}

\newcommand{\circuit}[3]{
        \begin{figure}[H]
            \centering
            \input{src/#3}
            \caption{#1}
            \label{circuit:#2}
        \end{figure}
}



% TODO
\newcommand{\ausdruck}[1]{\todo[linecolor=green,backgroundcolor=green!25,bordercolor=green]{#1}}
\newcommand{\ergaenzungen}[1]{\todo[linecolor=blue,backgroundcolor=blue!25,bordercolor=blue]{#1}}
\newcommand{\hinweis}[1]{\todo[linecolor=red,backgroundcolor=red!25,bordercolor=red]{#1}}
\newcommand{\mytodo}[1]{\todo[inline]{#1}}

% Semantics
\newcommand{\Index}[2][\empty]{\ifthenelse{\equal{#1}{\empty}}{\index{#2}#2}{\index{#1}#2}}
\newcommand{\Fachbegriff}[2][\empty]{\ifthenelse{\equal{#1}{\empty}}{\textit{\Index{#2}}}{\textit{\Index[#1]{#2}}}}
\newcommand{\NeuerBegriff}[2][\empty]{\ifthenelse{\equal{#1}{\empty}}{\textbf{\Index{#2}}}{\textbf{\Index[#1]{#2}}}}

\newcommand{\Formelzeichen}[1]{\texttt{#1}}
\newcommand*{\Wert}[3]{\Formelzeichen{#1}~=~{#2}~{#3}}
\newcommand{\Frage}[1]{\subsubsection*{#1}}

\newcommand{\WE}[2]{\SI{#1}{#2}}